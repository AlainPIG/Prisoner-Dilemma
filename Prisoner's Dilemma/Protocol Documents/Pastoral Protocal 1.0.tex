\documentclass[20pt]{article}
\usepackage{xeCJK}

\title{Pastoral Protocal 1.0}
\begin{document}
\maketitle
\tableofcontents
\clearpage
\section{协议内容}\

本协议通过前两次操作来传递信息,并维护一个可靠的信誉系统。

具体来说,我们通过“合作”来表达开始协议或接受协议,用“背叛”来表示拒绝协议,所有不在协议中规定的行为均视为失信行为。信誉系统通过玩家的历史操作记录,
维护其信誉分,并根据信誉分来决定接受协议(开始协议的概率)。\\

\textbf{协议代码100} \\
双方第一回合均合作,视为双方已经完成协议握手,在后续所有操作中合作。
双方均得到较多信誉分奖励。\\

\textbf{协议代码101} \\
第一回合一方合作(发起方),一方背叛。第二回合发起方背叛等待协议握手,
另一方合作接受协议。双方在后续所有操作中合作。发起方获得较多信誉分奖励,
接收方获得较少信誉分奖励。\\

\textbf{协议代码302} \\
双方第一回合均背叛,视为拒绝协议。\\

\textbf{协议代码402} \\
第一回合一方合作(发起方),一方背叛。第二回合均背叛,视作拒绝协议。
发起方获得较多信誉分奖励。\\

\textbf{协议代码501} \\
双方接受协议后任意一方背叛,视作协议终止。背叛方受到大量信誉分惩罚(如果
双方均背叛,则同时受到信誉分惩罚。\\

\section{协议算法}\

信誉分的计算方法为:
\begin{equation}
    s(n) = (s(n-1) + \alpha) \cdot \zeta / \lambda
\end{equation}

其中$s(n)$为第$n$次行动后的信誉分(取值范围在$0$到$1$之间),$\alpha$为发起(接受)协议的奖励分,$\zeta$为背叛协议的惩罚系数,$\lambda$为信誉分的自然衰减系数。

发起(接受)协议的概率有对方的分数决定,建议使用一下公式来计算:
\begin{equation}
    p = s ^ \beta 
\end{equation}

其中$\beta$为可调的参数。

\section{协议代码实现}\

只需在代码中加入通过历史对战记录计算信誉值得模块即可,实现起来较为简单。

\end{document}